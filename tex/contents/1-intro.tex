%-----------------------------------------------------------------
%	INTRODUCTION
%	!TEX root = ./../main.tex
%-----------------------------------------------------------------
\section{Introduction}
\subsection{Summary and goals}

The arrival of Data Science has caused a great impact in many different topics and professions. From every workplace we are seeing a huge effort to adapt the usual workflow into a new, more data-driven decision making environment, in which a proper data analysis can determine the fate of any kind of professional or academic endeavour.

One of the ecosystems in which Data Science is landing and leading a new path is in criminology. A whole new topic of study is emerging from the combination of criminology and applied statistics, often called \emph{Crime Analysis}; and this topic is starting to be one of the most important tools in the decision making processes in many Public Safety Bureaus. Specifically in the United States is possible to notice an important effort from several different cities to build open data portals in which they publish police records, arrest records, distress calls, etc.

Despite this gigantic effort in organising and publishing large amounts of data, we noticed how heterogeneous these data bases were from one city to another, and how messy and unclear could be for a crime analyst to get insights from this data. For every city you can find many different data warehouses, and everyone of them stores data from various sources in some specific and usually different static structures and categories. This situation might have been sufficient in its early stages, when data were small enough, but it can get utterly messy when trying to work with data from different cities and sky-rocket the budget of a crime analysis department. This problem is not only about crime analysis departments, it is a generalised concern on \emph{Big Data}.

The \emph{Data Hub} is emerging as one of the major solutions for these Big Data issues. A stable Data Hub alleviates the costs and complications of exploratory analysis, empowering non-technical users to exploit the possibilities of data intelligence.

Our proposal in order to make this situation simpler and clearer in \emph{Crime Data Analysis} is to build a national \emph{Crime Data Hub}, in which any Police Department or any agency concerned about public safety could enter and explore uncomplicated data.

For this \emph{proof of concept} we aim to present an initial stage of what an stable Crime Data Hub could be. A Data Hub that gathers, processes, normalises, and standardises crime data from several different cities around the United States for any analysis purpose. And more importantly, this has to be done working around the computational difficulties and limitations of dealing with huge data sets.

\bigskip
Our implementation is hosted on \url{https://aldomann.shinyapps.io/crimes-hub/}, and the code and scripts developed for this project can be found on \url{https://github.com/aldomann/open-data-project}, all released under the GNU General Public License v3.0.
